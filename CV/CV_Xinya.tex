% resume.tex
% (c) 2011 Arun Chaganty <arunchaganty@gmail.com>
% http://arun.chagantys.org/

% You can add the option 'web'

\documentclass[letterpaper,12pt]{resume}

\author{Xinya Du}
\website{http://sjtudxy.github.io/}
\email{louis.xyd@gmail.com}
\address{F1203006, 800 DC Rd, Shanghai, P.R. China, 200240}
%\phone{+86 18817865230}

\usepackage{ragged2e}

% I like circles instead of - 
\renewcommand{\labelitemii}{$\circ$}

\begin{document}
\resumeheader

%\section{Objective}

%Graduate school admission in a doctoral programƒme in the field of Computer
%Science and Engineering, followed by a career in research and teaching.
%\begin{itemize}
%  \item {\bf Research interests:} Natural Language Processing, Machine
%    Learning, Algorithmic Game Theory.
%\end{itemize} % End Objectives list

\section{Areas of Interest}

\begin{itemize}
\item Natural Language Processing and Machine Learning
\item Algorithmic Game Theory% and its applications in online resources allocation
\end {itemize}
 % End Objectives list



\section{Education}
\begin{itemize}
   \credential
    {\href{http://en.sjtu.edu.cn/}{Shanghai Jiao Tong University}}
    {Shanghai, China}
    {\href{http://english.seiee.sjtu.edu.cn/}{School of Electronic Information and Electrical Engineering}}
    {\textit{Sep. 2013 -- Jul. 2016 (expected)}}
    { 
    \begin{itemize}
      \item{B.S., Computer Science}%Major: \textbf{Computer Science}}
      \item{Overall GPA(141 credits): 94.50/100, \textbf{Rank: 1/122}}
      \item{Good \textbf{mathematics} training: All 11 math courses are above A and 10 of them are over A+.}
     % \item{Major GPA: 94.91/100, \textbf{Rank: 1/122}}
      \end{itemize}
    }
     \credential
    {\href{http://en.sjtu.edu.cn/}{Shanghai Jiao Tong University}}
   {Shanghai, China}
    {\href{http://speit.sjtu.edu.cn/indexen.html}{SJTU-ParisTech Elite Institute of Technology (SPEIT)}}
    {\textit{Sep. 2012 -- Jul. 2013}}
    { 
    \begin{itemize}
      \item{Overall GPA(77.5 credits): 92.85/100, \textbf{Rank: 1/62}}
      \end{itemize}
    }
\end{itemize}

\section{Research Experience}
%\hfill \url{http://xinyadu.weebly.com/research}}
\begin{itemize}
 
   \credential
   % {Deep Learning for Fine-grained Opinion Extraction}{Ithaca, NY, USA}
     {Neural Networks Approach for Fine-grained Opinion Extraction}{Ithaca, NY, USA}
    {NLP group, Cornell University}
    {\textit{Aug. 2015 -- Dec. 2015}}
    {
    \begin{itemize}
     \item{ \textit{Advisor: Claire Cardie}}
    % \item{ Improving and implementing deep learning algorithms to do fine-grained opinion extraction. }
    
    \item{Designed heuristic rules using dependency parse tree to eliminate inappropriate opinion candidates during inference. Empirical results showed higher precision and higher F-measure.}
    
    \item{Proposed to use deep recurrent neural networks to produce the \textit{n-best} labeling sequences, which can be fed into integer linear programming (ILP) system for joint inference.}
    
	 \item{Proposed to evaluate labeling sequences using \textbf{Sentence-Level Log-Likelihood (SLL)} at the output layer of deep recurrent neural networks. Empirical results showed over \textbf{5\%} improvement on F-measure of opinion target extraction and accurate  ranking for \textit{n-best} labeling sequences. }%
	 
	 \item{ Building an ensemble system using the above algorithm to participate in Belief and Sentiment Evaluation 2015 (\textbf{BeSt2015}).}
	 
%	   \item{ \textbf{Technical Report}: \textit{\footnotesize \url{http://xinyadu.weebly.com/uploads/6/6/3/5/66350551/techreport.pdf}}}%{\textit{link}}}
	
    \end{itemize}
    }
    
      \credential
    {Online Auction Mechanism Design with Time-varying Value}{Shanghai, China}
    {Shanghai Key Laboratory of Scalable Computing and Systems}{\textit{Jun. 2014 -- Jun. 2015}}
    {
    \begin{itemize}
    \item{ \textit{Advisor: Fan Wu}}
     \item{ Proposed the new scenario in online auction mechanism design where bidder's value may vary over time.}
      \item{
	Extended the classic payment determination algorithm (Myerson) to fit the new model. Proposed mechanism ensured strategy-proofness and achieved constant competitive ratio. 
	%\textbf{Paper is finished and submitted to IJCAI'16}.
        }
  
        \item{ \textbf{Technical Paper}: \textit{\footnotesize \url{http://sjtudxy.github.io/papers/techPaper.pdf}}}%{\textit{link}}}
       
    \end{itemize}
    }
    
     \credential
    {User Demographics Mining from Massive SMS Dataset}{Shanghai, China}
    {IIOT research center, Shanghai Jiao Tong University}
    {\textit{Jun. 2015 -- Sep. 2015}}
    {
    \begin{itemize}
    \item{ \textit{Advisor: Xinbing Wang}}
      
	 \item{Processed a massive and noisy SMS dataset, designed a filter to reduce noise and finally created a gold-standard dataset regarding user demographics with clustering method.}
	 \item{Built a system to do classification on the demographics information of users for whom we can provide targeted advertisements. }
     %\item{Code and report: \textit{link}}
    \end{itemize}
    }


\end{itemize}  % End Experience list

%%%%%%%%%%%%%%%%%%%%%%
\pagebreak
%%%%%%%%%%%%%%%%%%%%%%





\section{Awards \& Honors}
\begin{itemize}
\item {\bf National Scholarship 2013} {\footnotesize Highest honor for undergraduates in China, awarded to top 1\% students}

\item {\bf Shanghai Undergraduate Mathematical Contest in Modeling 2014} {\footnotesize The Second Prize}
    
   \item {\bf Shanghai Jiao Tong University Excellent Student Scholarship 2014}
   
    \item {\bf National Olympiad in Mathematics in Provinces 2011} {\footnotesize The First Prize }
    \item {\bf National Olympiad in Chemistry in Provinces 2011}
    {\footnotesize The First Prize}
  \item {\bf National Olympiad in Informatics in Provinces 2010}
 {\footnotesize The First Prize }
   


 \end{itemize}  % End Achievements list
%%%%%%%%%%%%%%%%%%%%%%
\section{Selected Course Projects \hfill \url{https://github.com/sjtudxy}}
\begin{itemize}
  \credential
    {\href{https://github.com/sjtudxy/SmallC-Compilier}{SmallC Compiler}}{Shanghai, China}
    {Shanghai Jiao Tong Univeristy}{\textit{Oct. 2014 - Dec. 2014}}
    { 
    \begin{itemize}
      \item{ 
      Built a compiler translating the basic C source code to MIPS assembly code with Lex and Yacc.
        }
       \item{Top performance in testing, Score: 100/100}
    \end{itemize}
    }
    
      \credential
    {{Design and Implementation of File System}}{Shanghai, China}
    {Shanghai Jiao Tong Univeristy}{\textit{May. 2014 - Jun. 2014}}
    { 
    \begin{itemize}
      \item{ 
     Designed and implemented a basic disk-like secondary storage server as well as a basic file system to act as a client, which uses the disk services provided by the server. }
       \item{Top performance in testing, Score: 100/100}
    \end{itemize}
    }
   
\end{itemize}  % End Experience list




%%%%%%%%%%%%%%%%%%%%%%
\section{Skills}
\begin{itemize}
\item{ \bf Natural Languages:} {Chinese (mother tongue), English (fluent), French (basic)}
  \item {\bf Programming Languages:} {  C/C++, Python, Java,
   {\LaTeX}, Matlab, Mathematica }
  \item {\bf Operating Systems:} Linux, Mac OS X, Windows
   \item {\bf Software:}{ Stanford CoreNLP,  Eclipse, VisualStudio SDK, Lex, Yacc
   
%   GTK+, wxWidgets,
%    Django, Win32, Windows Device Framework (WDF), WPF, Visual Studio
%    SDK, Z3, Infer.NET
    }  
\end{itemize} % End Skills list

%%%%%%%%%%%%%%%%%%%%%%
\section{Volunteering Activities}
\begin{itemize}
  
   \item {Volunteer of Shanghai Library} {\hfill \textit{Aug. 2014}}
   
    \item {Secretary of Team of Youth Volunteers, SJTU}{\hfill \textit{Sep. 2012 - Jul. 2013}} 
    
     \item {Volunteer of Asia Student Supercomputer Challenge (ASC)} {\hfill \textit{May. 2013}}
     
       \item {Voluntary blood donation} {\hfill \textit{Apr. 2013}}
   
  
\end{itemize} 

\end{document}

